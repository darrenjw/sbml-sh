\documentclass[11pt,a4paper]{article}
\usepackage{times,url,hyperref}

\topmargin=-0.5in
\oddsidemargin=0in
\evensidemargin=0in
\textwidth=6.5in
\textheight=9.5in

\title{SBML-shorthand specification, Version 3.1.1}
\author{\href{http://www.staff.ncl.ac.uk/d.j.wilkinson/}{Darren J
Wilkinson}\\
School of Mathematics and Statistics\\
University of Newcastle upon Tyne, UK\\
email: \href{mailto:d.j.wilkinson@ncl.ac.uk}{\texttt{d.j.wilkinson@ncl.ac.uk}}}
\date{\today}

\begin{document}
\sf
\maketitle

\tableofcontents

\newpage

\section{Introduction}

\subsection{Overview}

SBML-shorthand is a language for describing systems biology models. It
is designed primarily for translation (compilation) into SBML. It is
not associated with any particular modelling or simulation tool. SBML
itself is not intended to be read by humans or written ``by hand'' ---
SBML-shorthand is. Note that SBML-shorthand is not intended to cover
every feature in SBML --- just the most commonly used aspects.

\subsection{Version}

The SBML-shorthand specification has a version number of the form
$a.b.c$. The $a$ refers to the SBML Level that the shorthand is a
target for. The $b$ refers to the SBML Version that the shorthand is a
target for. The $c$ refers to the version number of the SBML-shorthand
specification for that particular SBML Level/Version combination. So,
3.1.1 is the first draft of a SBML-shorthand specification aimed at
SBML Level 3, Version 1 (core).

\subsection{Example}

The following text is a complete SBML-shorthand description of a
simple model for Michaelis-Menten enzyme kinetics.

{\small
\begin{verbatim}
@model:3.3.1=MichaelisMentenKinetics "Michaelis-Menten Kinetics"
@compartments
 cell=1
@species
 cell:Substrate=1000
 cell:Enzyme=100
 cell:Complex=0
 cell:Product=0
@parameters
 k1=1
 k1r=2
@reactions
@rr=SubstrateEnzymeBinding "Substrate-enzyme binding"
 Substrate+Enzyme -> Complex
 k1*Substrate*Enzyme-k1r*Complex
@r=Conversion
 Complex -> Product + Enzyme
 k2*Complex : k2=3
\end{verbatim}}

\noindent The ``compiled'' SBML corresponding to this shorthand description is
given below.

{\scriptsize
\begin{verbatim}
<?xml version="1.0" encoding="UTF-8"?>
<sbml xmlns="http://www.sbml.org/sbml/level3/version1/core" level="3" version="1">
  <model id="MichaelisMentenKinetics" name="Michaelis-Menten Kinetics">
    <listOfCompartments>
      <compartment id="cell" size="1" constant="true"/>
    </listOfCompartments>
    <listOfSpecies>
      <species id="Substrate" compartment="cell" initialAmount="1000" hasOnlySubstanceUnits="false"
               boundaryCondition="false" constant="false"/>
      <species id="Enzyme" compartment="cell" initialAmount="100" hasOnlySubstanceUnits="false" 
               boundaryCondition="false" constant="false"/>
      <species id="Complex" compartment="cell" initialAmount="0" hasOnlySubstanceUnits="false" 
               boundaryCondition="false" constant="false"/>
      <species id="Product" compartment="cell" initialAmount="0" hasOnlySubstanceUnits="false"
               boundaryCondition="false" constant="false"/>
    </listOfSpecies>
    <listOfParameters>
      <parameter id="k1" value="1" constant="true"/>
      <parameter id="k1r" value="2" constant="true"/>
    </listOfParameters>
    <listOfReactions>
      <reaction id="SubstrateEnzymeBinding" name="Substrate-enzyme binding" reversible="true" fast="false">
        <listOfReactants>
          <speciesReference species="Substrate" stoichiometry="1" constant="false"/>
          <speciesReference species="Enzyme" stoichiometry="1" constant="false"/>
        </listOfReactants>
        <listOfProducts>
          <speciesReference species="Complex" stoichiometry="1" constant="false"/>
        </listOfProducts>
        <kineticLaw>
          <math xmlns="http://www.w3.org/1998/Math/MathML">
            <apply>
              <minus/>
              <apply>
                <times/>
                <ci> k1 </ci>
                <ci> Substrate </ci>
                <ci> Enzyme </ci>
              </apply>
              <apply>
                <times/>
                <ci> k1r </ci>
                <ci> Complex </ci>
              </apply>
            </apply>
          </math>
        </kineticLaw>
      </reaction>
      <reaction id="Conversion" reversible="false" fast="false">
        <listOfReactants>
          <speciesReference species="Complex" stoichiometry="1" constant="false"/>
        </listOfReactants>
        <listOfProducts>
          <speciesReference species="Product" stoichiometry="1" constant="false"/>
          <speciesReference species="Enzyme" stoichiometry="1" constant="false"/>
        </listOfProducts>
        <kineticLaw>
          <math xmlns="http://www.w3.org/1998/Math/MathML">
            <apply>
              <times/>
              <ci> k2 </ci>
              <ci> Complex </ci>
            </apply>
          </math>
          <listOfLocalParameters>
            <localParameter id="k2" value="3"/>
          </listOfLocalParameters>
        </kineticLaw>
      </reaction>
    </listOfReactions>
  </model>
</sbml>
\end{verbatim}}

It is hopefully immediately transparent that the SBML-shorthand
notation is much easier to read and write than the corresponding SBML,
but that there is a simple one-one correspondence between the model
elements.

\section{Specification}

\subsection{Basic structure}

The description format is plain ASCII text. The suggested file
extension is \verb$.mod$, but this is not required. All whitespace
other than carriage returns is insignificant (unless it is contained
within a quoted ``name'' element). Carriage returns are
significant. The description is case-sensitive. Blank lines are
ignored. The comment character is \verb$#$ - all text from a \verb$#$
to the end of the line is ignored.

\subsection{Model}

The model description must begin with the characters
\verb$@model:3.3.1=$ (the 3.3.1 corresponds to the version number of
this specification). The text following the \verb$=$ on the first line
is the model id. An optional model name may also be specified, following
the id, enclosed in double-quotes. 

The model line may be followed with an additional line specifying
global model units. For example, the SBML-shorthand

{\small
\begin{verbatim}
@model:3.3.1=mymodel "My model"
 s=substance, t=second, v=litre
\end{verbatim}
}

\noindent would be translated to

{\small
\begin{verbatim}
  <model id="MyName" name="My Model" substanceUnits="substance" 
                     timeUnits="second" volumeUnits="litre">
\end{verbatim}
}

The model attributes \verb$substanceUnits$, \verb$timeUnits$,
\verb$volumeUnits$, \verb$areaUnits$, \verb$lengthUnits$,
\verb$extentUnits$, and \verb$conversionFactor$ are represented by the
letters \verb$s$, \verb$t$, \verb$v$, \verb$a$, \verb$l$, \verb$e$ and
\verb$c$, respectively.

The model is completed with the
specification of the 5 sections, \verb$@units$, \verb$@compartments$,
\verb$@species$,
\verb$@parameters$, \verb$@rules$, \verb$@reactions$ and \verb$@events$, corresponding to the SBML
sections, \verb$<listOfUnitDefinitions>$, \verb$<listOfCompartments>$, \verb$<listOfSpecies>$,
\verb$<listOfParameters>$, \verb$<listOfRules>$, \verb$<listOfReactions>$ and \verb$<listOfEvents>$,
respectively. The sections must occur in the stated order. Sections
are optional, but if present, may not be empty. These are the only
sections covered by this specification. Subsequent specifications will
contain additional sections. 

\subsection{Units}

The format of the individual sections will be explained mainly by
example. The following SBML-shorthand

{\small
\begin{verbatim}
@units
 substance=item
 mmls=mole:s=-3; litre:e=-1; second:e=-1
\end{verbatim}}

\noindent would be translated to

{\small
\begin{verbatim}
    <listOfUnitDefinitions>
      <unitDefinition id="substance" name="item (substance default)">
        <listOfUnits>
          <unit kind="item" exponent="1" scale="0" multiplier="1"/>
        </listOfUnits>
      </unitDefinition>
      <unitDefinition id="mmls">
        <listOfUnits>
          <unit kind="mole" exponent="1" scale="-3" multiplier="1"/>
          <unit kind="litre" exponent="-1" scale="0" multiplier="1"/>
          <unit kind="second" exponent="-1" scale="0" multiplier="1"/>
        </listOfUnits>
      </unitDefinition>
    </listOfUnitDefinitions>
\end{verbatim}}

The unit attributes \verb$exponent$, \verb$multiplier$, and \verb$scale$
are denoted by the letters \verb$e$, \verb$m$ and
\verb$s$, respectively. Note that as there is (currently)
no way to refer to units elsewhere in SBML-shorthand, the only
function for this section is to re-define the globally defined units
declared in the model section.

\subsection{Compartments}


The following SBML-shorthand

{\small
\begin{verbatim}
@compartments
 cell=1
 cytoplasm=0.8
 nucleus=0.1
 mito "Mitochondria"
 cell2
\end{verbatim}}

\noindent would be translated to

{\small
\begin{verbatim}
    <listOfCompartments>
      <compartment id="cell" size="1" constant="true"/>
      <compartment id="cytoplasm" size="0.8" constant="true"/>
      <compartment id="nucleus" size="0.1" constant="true"/>
      <compartment id="mito" name="Mitochondria" constant="true"/>
      <compartment id="cell2" constant="true"/>
    </listOfCompartments>
\end{verbatim}}

\noindent Note that if a \verb$name$ attribute is to be specified, it should be
specified at the end of the line in double-quotes. This is true for
other SBML elements too.

\subsection{Species}

The following shorthand

{\small
\begin{verbatim}
@species
 cell:Gene = 10b "The Gene"
 cell:P2=0
 cell:S1=100 s
 cell:[S2]=20 sc
 cell:[S3]=1000 bc
 mito:S4=0 b
\end{verbatim}}

\noindent would be translated to

{\scriptsize
\begin{verbatim}
    <listOfSpecies>
      <species id="Gene" name="The Gene" compartment="cell" initialAmount="10"
               hasOnlySubstanceUnits="false" boundaryCondition="true" constant="false"/>
      <species id="P2" compartment="cell" initialAmount="0"
               hasOnlySubstanceUnits="false" boundaryCondition="false" constant="false"/>
      <species id="S1" compartment="cell" initialAmount="100"
               hasOnlySubstanceUnits="true" boundaryCondition="false" constant="false"/>
      <species id="S2" compartment="cell" initialConcentration="20"
               hasOnlySubstanceUnits="true" boundaryCondition="false" constant="true"/>
      <species id="S3" compartment="cell" initialConcentration="1000"
               hasOnlySubstanceUnits="false" boundaryCondition="true" constant="true"/>
      <species id="S4" compartment="mito" initialAmount="0"
               hasOnlySubstanceUnits="false" boundaryCondition="true" constant="false"/>
    </listOfSpecies>
\end{verbatim}}

Compartments are compulsory. An \verb$initialConcentration$ (as opposed
to an \verb$initialAmount$) is flagged by enclosing the species \verb$id$ in
brackets. The boolean attributes \verb$hasOnlySubstanceUnits$,
\verb$boundaryCondition$ and \verb$constant$ can be set to \verb$true$
by appending the letters \verb$s$, \verb$b$ and \verb$c$,
respectively. The order of the flags is not important.

\subsection{Parameters}

The section

{\small
\begin{verbatim}
@parameters
 k1=1  # a comment
 k2 = 10v "K2"
\end{verbatim}}

\noindent would be translated to

{\small
\begin{verbatim}
    <listOfParameters>
      <parameter id="k1" value="1" constant="true"/>
      <parameter id="k2" name="K2" value="10" constant="false"/>
    </listOfParameters>
\end{verbatim}}

\subsection{Rules}

Currently only assignment and rate rules are supported. Assignment
rules are supported using a fairly
obvious notation. Rate rules can be specified by prefixing the rule
with \verb$@rate:$. So

{\small
\begin{verbatim}
@rules
 PTot = P + 2*P2
 v = 1 + 0.5*t
 @rate:v2=1
\end{verbatim}
}

would be translated to    

{\small
\begin{verbatim}
    <listOfRules>
      <assignmentRule variable="PTot">
        <math xmlns="http://www.w3.org/1998/Math/MathML">
          <apply>
            <plus/>
            <ci> P </ci>
            <apply>
              <times/>
              <cn type="integer"> 2 </cn>
              <ci> P2 </ci>
            </apply>
          </apply>
        </math>
      </assignmentRule>
      <assignmentRule variable="v">
        <math xmlns="http://www.w3.org/1998/Math/MathML">
          <apply>
            <plus/>
            <cn type="integer"> 1 </cn>
            <apply>
              <times/>
              <cn> 0.5 </cn>
              <csymbol encoding="text" 
                       definitionURL="http://www.sbml.org/sbml/symbols/time">
                       t </csymbol>
            </apply>
          </apply>
        </math>
      </assignmentRule>
      <rateRule variable="v2">
        <math xmlns="http://www.w3.org/1998/Math/MathML">
          <cn type="integer"> 1 </cn>
        </math>
      </rateRule>
    </listOfRules>
\end{verbatim}
}



\subsection{Reactions}

Each reaction is specified by exactly two or three lines of text. The first
line declares the reaction name and whether the reaction is reversible
(\verb$@rr=$ for reversible and \verb$@r=$ otherwise). The second line
specifies the reaction itself using a fairly standard notation. If a
modifier is to be included, it should occur at the end, separated from
the rest of the reaction by a~\verb$:$. The (optional)
third line specifies the full rate law for the kinetics. If local
parameters are used, they should be declared on the same line in a
comma-separated list, separated from the rate law using a~\verb$:$

So, for example,

{\small
\begin{verbatim}
@reactions
@r=RepressionBinding "Repression Binding"
 Gene + 2P -> P2Gene
 k2*Gene
@rr=Reverse
 P2Gene -> Gene+2P
 k1r*P2Gene : k1r=1,k2=3
@r=NoKL
 Harry->Jim : Fred
@r=Test
 Fred -> Fred2
 k4*Fred : k4=1

\end{verbatim}}

\noindent would translate to

{\scriptsize
\begin{verbatim}
    <listOfReactions>
      <reaction id="RepressionBinding" name="Repression Binding" reversible="false" fast="false">
        <listOfReactants>
          <speciesReference species="Gene" stoichiometry="1" constant="false"/>
          <speciesReference species="P" stoichiometry="2" constant="false"/>
        </listOfReactants>
        <listOfProducts>
          <speciesReference species="P2Gene" stoichiometry="1" constant="false"/>
        </listOfProducts>
        <kineticLaw>
          <math xmlns="http://www.w3.org/1998/Math/MathML">
            <apply>
              <times/>
              <ci> k2 </ci>
              <ci> Gene </ci>
            </apply>
          </math>
        </kineticLaw>
      </reaction>
      <reaction id="Reverse" reversible="true" fast="false">
        <listOfReactants>
          <speciesReference species="P2Gene" stoichiometry="1" constant="false"/>
        </listOfReactants>
        <listOfProducts>
          <speciesReference species="Gene" stoichiometry="1" constant="false"/>
          <speciesReference species="P" stoichiometry="2" constant="false"/>
        </listOfProducts>
        <kineticLaw>
          <math xmlns="http://www.w3.org/1998/Math/MathML">
            <apply>
              <times/>
              <ci> k1r </ci>
              <ci> P2Gene </ci>
            </apply>
          </math>
          <listOfLocalParameters>
            <localParameter id="k1r" value="1"/>
            <localParameter id="k2" value="3"/>
          </listOfLocalParameters>
        </kineticLaw>
      </reaction>
      <reaction id="NoKL" reversible="false" fast="false">
        <listOfReactants>
          <speciesReference species="Harry" stoichiometry="1" constant="false"/>
        </listOfReactants>
        <listOfProducts>
          <speciesReference species="Jim" stoichiometry="1" constant="false"/>
        </listOfProducts>
        <listOfModifiers>
          <modifierSpeciesReference species="Fred"/>
        </listOfModifiers>
      </reaction>
      <reaction id="Test" reversible="false" fast="false">
        <listOfReactants>
          <speciesReference species="Fred" stoichiometry="1" constant="false"/>
        </listOfReactants>
        <listOfProducts>
          <speciesReference species="Fred2" stoichiometry="1" constant="false"/>
        </listOfProducts>
        <kineticLaw>
          <math xmlns="http://www.w3.org/1998/Math/MathML">
            <apply>
              <times/>
              <ci> k4 </ci>
              <ci> Fred </ci>
            </apply>
          </math>
          <listOfLocalParameters>
            <localParameter id="k4" value="1"/>
          </listOfLocalParameters>
        </kineticLaw>
      </reaction>
    </listOfReactions>
\end{verbatim}}

\subsection{Events}

The following SBML-shorthand
\begin{verbatim}
@events
 reset= t>=25 : P=100;P2=0 "Reset event"
 flush= P>10 ; 20 : P=0
\end{verbatim}
would be translated to

{\scriptsize
\begin{verbatim}
    <listOfEvents>
      <event id="reset" name="Reset event">
        <trigger>
          <math xmlns="http://www.w3.org/1998/Math/MathML">
            <apply>
              <geq/>
              <csymbol encoding="text" definitionURL="http://www.sbml.org/sbml/symbols/time"> t </csymbol>
              <cn type="integer"> 25 </cn>
            </apply>
          </math>
        </trigger>
        <listOfEventAssignments>
          <eventAssignment variable="P">
            <math xmlns="http://www.w3.org/1998/Math/MathML">
              <cn type="integer"> 100 </cn>
            </math>
          </eventAssignment>
          <eventAssignment variable="P2">
            <math xmlns="http://www.w3.org/1998/Math/MathML">
              <cn type="integer"> 0 </cn>
            </math>
          </eventAssignment>
        </listOfEventAssignments>
      </event>
      <event id="flush" useValuesFromTriggerTime="true">
        <trigger>
          <math xmlns="http://www.w3.org/1998/Math/MathML">
            <apply>
              <gt/>
              <ci> P </ci>
              <cn type="integer"> 10 </cn>
            </apply>
          </math>
        </trigger>
        <delay>
          <math xmlns="http://www.w3.org/1998/Math/MathML">
            <cn type="integer"> 20 </cn>
          </math>
        </delay>
        <listOfEventAssignments>
          <eventAssignment variable="P">
            <math xmlns="http://www.w3.org/1998/Math/MathML">
              <cn type="integer"> 0 </cn>
            </math>
          </eventAssignment>
        </listOfEventAssignments>
      </event>
    </listOfEvents>
\end{verbatim}
}

\noindent So, the event assignments occur in a semi-colon-separated list after a
colon, and the optional delay element can be specified after the
trigger condition, separated from it with a semi-colon. If the variable
\verb$t$ occurs in the trigger condition, it is interpreted as the
SBML time \texttt{csymbol}.



\section{Tool support}

The author has written a reference implementation of a SBML-shorthand
to SBML compiler, written in Python. This can be
used either as a simple command-line translator, or by Python
programmers as a Python module. Documentation for this module is
available elsewhere. The Python module requires libSBML (and requires that
libSBML is built with Python bindings).
A reference implementation of a SBML to SBML-shorthand translator is
also available.

See the SBML-shorthand web site for further details.

\section{Acknowledgements}

Several people have contributed to SBML-shorthand and the associated
translation tools, including Jeremy Purvis, Carole Proctor, Mark
Muldoon, and Lukas Endler.


\section{Links}

\begin{tabular}{rl}
SBML-shorthand & {\small
\href{http://www.staff.ncl.ac.uk/d.j.wilkinson/software/sbml-sh/}{\url{www.staff.ncl.ac.uk/d.j.wilkinson/software/sbml-sh/}}
}\\
SBML & \href{http://sbml.org/}{\url{sbml.org}} \\
Python & \href{http://www.python.org/}{\url{www.python.org}}
\end{tabular}



\end{document}
